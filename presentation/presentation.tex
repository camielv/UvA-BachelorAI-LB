\documentclass{beamer}

\mode<presentation>
{
  \usetheme{Berlin}
  \usecolortheme{default}
  \setbeamercovered{transparent}
}

\usepackage[english]{babel}
\usepackage[all]{xy}
\usepackage[latin1]{inputenc}

\usepackage{times}
\usepackage[T1]{fontenc}


\title[Sentiment Analysis]
{Sentiment Analysis}

\subtitle{A Probabilistic Approach}

\author[Gieske, Laan, Ten Velthuis, Verschoor, Wiggers] % (optional, use only with lots of authors)
{S.~A.~Gieske \and S.~Laan \and D.~S.~Ten Velthuis \and C.~R.~Verschoor \and A.~J.~Wiggers}

\institute[University of Amsterdam] % (optional, but mostly needed)
{
  Faculty of Science (FNWI) \\
  University of Amsterdam
}

\AtBeginSubsection[]
{
  \begin{frame}<beamer>{Outline}
    \tableofcontents[currentsection,currentsubsection]
  \end{frame}
}

\begin{document}

\begin{frame}
  \titlepage
\end{frame}

\begin{frame}{Outline}
  \tableofcontents
\end{frame}


\section{The goal}

\begin{frame}{The goal of the project}
\begin{block}{Project Description}
Human operator `becomes' a robot:
\end{block}
\begin{itemize}
\item Mimic head through VR glasses
\item Mimic upper body through Kinect
\item Control lower body through gestures
\end{itemize}
\end{frame}

\section{Motivation}

\begin{frame}{Motivation}
We think this project is interesting because:\\
\begin{itemize}
\item<2-> Relatively new
\item<3-> Step closer to human interface.
\item<4-> Requires interesting problems to be solved.
\end{itemize}
\end{frame}

\section{Solved problems}

\subsection{Multiple programming languages}
\begin{frame}[t]{VR glasses}
\begin{block}Several languages have to be used for the different hardware.
\end{block}
\pause
\begin{block}{Solution}
A TCP server with
\begin{itemize}
\item The nao as server, sends images
\item The VR glasses program as client, sends joint values
\item The Kinect program as client, sends joint values
\end{itemize}
\end{block}
\end{frame}

\subsection{VR glasses}
\begin{frame}[t]{VR glasses}
\begin{block}Using the built-in gyroscope, calculate joint angles for the Nao.
\end{block}
\pause
\begin{block}{Solution}
\xymatrix{ \textnormal{Find angles} \ar[d] \\
       \textnormal{Send through TCP} \ar[d]\\
    \textnormal{Set angles} }

\end{block}
\end{frame}

\section{Problems and Possible Solutions}
\subsection{Inverse Kinematics}
\begin{frame}[t]{Inverse Kinematics}
\begin{block}Given a desired position for the robot, calculate joint angles.
\end{block}
\pause
\begin{block}{Possible Solution}
\begin{itemize}

\end{itemize}
\end{block}
\end{frame}

\subsection{Balance}
\begin{frame}[t]{Robots and Balance}
\begin{block}{Problem}
Nao is easily brought out of balance. Human walking is difficult.
\end{block}
\pause
\begin{block}{Possible Solution}
Either:
\begin{itemize}
\item Use gestures 
\item Create autobalancing (Nao built-in does not work)
\end{itemize}
\end{block}
\end{frame}


\section{Planning}
\subsection{Tasks}
\begin{frame}{Subtask Graph}
\footnotesize
\xymatrix{
& \textnormal{Project}\ar@/_/[dl] \ar[d] \ar@/^/[dr]   & \\
\textnormal{VR glasses}\ar@/_/[ddr] & \textnormal{Kinect to pc}\ar[d]                & \textnormal{Balance Nao}\ar[dd]   \\
&\textnormal{Tutorials OpenNI}\ar[d] \\
&\textnormal{Wireframe to Nao} \ar[d] & \textnormal{Autobalance during motion} \ar@/^/[dl] \\
                                                               & \textnormal{Smooth motions, gestures} \\
}
\normalsize

\end{frame}

\subsection{Time Planning}

\begin{frame}{Time Scheme}
\begin{tabular}{|c|l|c|}
\hline
Week & Activity & Team Member\\
\hline
1  & Get OpenNI to work       & All\\
  & Connect Kinect to PC       & All\\
  & Contact supervisor         & Duncan\\
\hline 
2   
  & Designing program structure     & All \\
  & Inverse kinematics         & All \\
  & TCP server\&client programs     & All \\
  & Extract wireframe from Kinect     & Steven\\
  & VR glasses gyroscope       & Duncan\\
  & Control Nao based on client messages   & Auke \\
  & Report & All \\
\hline\hline
  & Presentation \& Demo        & All\\
\hline
\end{tabular}
\end{frame}

\begin{frame}{Time Scheme continued}
\begin{tabular}{|c|l|c|}
\hline
3  & More inverse kinematics      & All\\
  & Gestures           & All\\
  & Autobalance nao         & Auke \\
\hline 
4   & Bug fixing           & All\\
  & Error margin         & All\\
  & Report           & All\\
\hline\hline
  & Endpresentation, final demo    & All \\
\hline
\end{tabular}
\end{frame}

\subsection{Error margin}
\begin{frame}{Uncertainties}
\begin{itemize}[<+->]
\item 
Balance issues, either:\begin{itemize}
\item Don't mimic the legs.
\item Let the balancing system override the mimicking system.
\item Use a gesture system for the lower part of the body. Probably this option.
\end{itemize}
\item Unforeseen technical difficulties
\end{itemize}
\end{frame}

\end{document}